% Options for packages loaded elsewhere
\PassOptionsToPackage{unicode}{hyperref}
\PassOptionsToPackage{hyphens}{url}
\documentclass[
]{article}
\usepackage{xcolor}
\usepackage[margin=1in]{geometry}
\usepackage{amsmath,amssymb}
\setcounter{secnumdepth}{-\maxdimen} % remove section numbering
\usepackage{iftex}
\ifPDFTeX
  \usepackage[T1]{fontenc}
  \usepackage[utf8]{inputenc}
  \usepackage{textcomp} % provide euro and other symbols
\else % if luatex or xetex
  \usepackage{unicode-math} % this also loads fontspec
  \defaultfontfeatures{Scale=MatchLowercase}
  \defaultfontfeatures[\rmfamily]{Ligatures=TeX,Scale=1}
\fi
\usepackage{lmodern}
\ifPDFTeX\else
  % xetex/luatex font selection
\fi
% Use upquote if available, for straight quotes in verbatim environments
\IfFileExists{upquote.sty}{\usepackage{upquote}}{}
\IfFileExists{microtype.sty}{% use microtype if available
  \usepackage[]{microtype}
  \UseMicrotypeSet[protrusion]{basicmath} % disable protrusion for tt fonts
}{}
\makeatletter
\@ifundefined{KOMAClassName}{% if non-KOMA class
  \IfFileExists{parskip.sty}{%
    \usepackage{parskip}
  }{% else
    \setlength{\parindent}{0pt}
    \setlength{\parskip}{6pt plus 2pt minus 1pt}}
}{% if KOMA class
  \KOMAoptions{parskip=half}}
\makeatother
\usepackage{color}
\usepackage{fancyvrb}
\newcommand{\VerbBar}{|}
\newcommand{\VERB}{\Verb[commandchars=\\\{\}]}
\DefineVerbatimEnvironment{Highlighting}{Verbatim}{commandchars=\\\{\}}
% Add ',fontsize=\small' for more characters per line
\usepackage{framed}
\definecolor{shadecolor}{RGB}{248,248,248}
\newenvironment{Shaded}{\begin{snugshade}}{\end{snugshade}}
\newcommand{\AlertTok}[1]{\textcolor[rgb]{0.94,0.16,0.16}{#1}}
\newcommand{\AnnotationTok}[1]{\textcolor[rgb]{0.56,0.35,0.01}{\textbf{\textit{#1}}}}
\newcommand{\AttributeTok}[1]{\textcolor[rgb]{0.13,0.29,0.53}{#1}}
\newcommand{\BaseNTok}[1]{\textcolor[rgb]{0.00,0.00,0.81}{#1}}
\newcommand{\BuiltInTok}[1]{#1}
\newcommand{\CharTok}[1]{\textcolor[rgb]{0.31,0.60,0.02}{#1}}
\newcommand{\CommentTok}[1]{\textcolor[rgb]{0.56,0.35,0.01}{\textit{#1}}}
\newcommand{\CommentVarTok}[1]{\textcolor[rgb]{0.56,0.35,0.01}{\textbf{\textit{#1}}}}
\newcommand{\ConstantTok}[1]{\textcolor[rgb]{0.56,0.35,0.01}{#1}}
\newcommand{\ControlFlowTok}[1]{\textcolor[rgb]{0.13,0.29,0.53}{\textbf{#1}}}
\newcommand{\DataTypeTok}[1]{\textcolor[rgb]{0.13,0.29,0.53}{#1}}
\newcommand{\DecValTok}[1]{\textcolor[rgb]{0.00,0.00,0.81}{#1}}
\newcommand{\DocumentationTok}[1]{\textcolor[rgb]{0.56,0.35,0.01}{\textbf{\textit{#1}}}}
\newcommand{\ErrorTok}[1]{\textcolor[rgb]{0.64,0.00,0.00}{\textbf{#1}}}
\newcommand{\ExtensionTok}[1]{#1}
\newcommand{\FloatTok}[1]{\textcolor[rgb]{0.00,0.00,0.81}{#1}}
\newcommand{\FunctionTok}[1]{\textcolor[rgb]{0.13,0.29,0.53}{\textbf{#1}}}
\newcommand{\ImportTok}[1]{#1}
\newcommand{\InformationTok}[1]{\textcolor[rgb]{0.56,0.35,0.01}{\textbf{\textit{#1}}}}
\newcommand{\KeywordTok}[1]{\textcolor[rgb]{0.13,0.29,0.53}{\textbf{#1}}}
\newcommand{\NormalTok}[1]{#1}
\newcommand{\OperatorTok}[1]{\textcolor[rgb]{0.81,0.36,0.00}{\textbf{#1}}}
\newcommand{\OtherTok}[1]{\textcolor[rgb]{0.56,0.35,0.01}{#1}}
\newcommand{\PreprocessorTok}[1]{\textcolor[rgb]{0.56,0.35,0.01}{\textit{#1}}}
\newcommand{\RegionMarkerTok}[1]{#1}
\newcommand{\SpecialCharTok}[1]{\textcolor[rgb]{0.81,0.36,0.00}{\textbf{#1}}}
\newcommand{\SpecialStringTok}[1]{\textcolor[rgb]{0.31,0.60,0.02}{#1}}
\newcommand{\StringTok}[1]{\textcolor[rgb]{0.31,0.60,0.02}{#1}}
\newcommand{\VariableTok}[1]{\textcolor[rgb]{0.00,0.00,0.00}{#1}}
\newcommand{\VerbatimStringTok}[1]{\textcolor[rgb]{0.31,0.60,0.02}{#1}}
\newcommand{\WarningTok}[1]{\textcolor[rgb]{0.56,0.35,0.01}{\textbf{\textit{#1}}}}
\usepackage{graphicx}
\makeatletter
\newsavebox\pandoc@box
\newcommand*\pandocbounded[1]{% scales image to fit in text height/width
  \sbox\pandoc@box{#1}%
  \Gscale@div\@tempa{\textheight}{\dimexpr\ht\pandoc@box+\dp\pandoc@box\relax}%
  \Gscale@div\@tempb{\linewidth}{\wd\pandoc@box}%
  \ifdim\@tempb\p@<\@tempa\p@\let\@tempa\@tempb\fi% select the smaller of both
  \ifdim\@tempa\p@<\p@\scalebox{\@tempa}{\usebox\pandoc@box}%
  \else\usebox{\pandoc@box}%
  \fi%
}
% Set default figure placement to htbp
\def\fps@figure{htbp}
\makeatother
\setlength{\emergencystretch}{3em} % prevent overfull lines
\providecommand{\tightlist}{%
  \setlength{\itemsep}{0pt}\setlength{\parskip}{0pt}}
\usepackage{bookmark}
\IfFileExists{xurl.sty}{\usepackage{xurl}}{} % add URL line breaks if available
\urlstyle{same}
\hypersetup{
  pdftitle={homework1},
  pdfauthor={xbw},
  hidelinks,
  pdfcreator={LaTeX via pandoc}}

\title{homework1}
\author{xbw}
\date{2025-09-17}

\begin{document}
\maketitle

\begin{enumerate}
\def\labelenumi{\arabic{enumi}.}
\tightlist
\item
\end{enumerate}

\begin{itemize}
\tightlist
\item
  a
\end{itemize}

\begin{Shaded}
\begin{Highlighting}[]
\NormalTok{ iowa.df}\OtherTok{\textless{}{-}}\FunctionTok{read.csv}\NormalTok{(}\StringTok{"D:/postgraduate/data science/mynotes/data/iowa.csv"}\NormalTok{, }\AttributeTok{sep =} \StringTok{\textquotesingle{};\textquotesingle{}}\NormalTok{, }\AttributeTok{header=}\NormalTok{T)}
\end{Highlighting}
\end{Shaded}

\begin{verbatim}
b.33 rows and 10 columns

c.Year Rain0 Temp1 Rain1 Temp2 Rain2 Temp3 Rain3 Temp4 Yield

d.  The value of row 5, column 7 of iowa.df is 79.7

e.  


``` r
iowa.df[2,]
```

```
##   Year Rain0 Temp1 Rain1 Temp2 Rain2 Temp3 Rain3 Temp4 Yield
## 2 1931 14.76  57.5  3.83    75  2.72  77.2   3.3  72.6  32.9
```
\end{verbatim}

\begin{enumerate}
\def\labelenumi{\arabic{enumi}.}
\setcounter{enumi}{1}
\item
  \begin{enumerate}
  \def\labelenumii{\alph{enumii}.}
  \tightlist
  \item
    因为vector1的元素被定义为character类型,所以vector1的元素在比较和排序时根据字符的ASCII码按照字典序进行,``5'',``12'',``7'',``32'',按照首个字符的ASCII码进行排序,``1''\textless{}``3''\textless{}``5''\textless{}``7'',所以max(vector1)返回``7'',sort(vector1)返回''12''
    ``32'' ``5'' ``7'' sum()的参数不能为character,所以会报错
  \item
  \end{enumerate}

  \begin{itemize}
  \tightlist
  \item
    当创建向量vector2 \textless-
    c(``5'',7,12)时,由于包含了不同类型的元素,所以R会自动进行强制类型转换,以确保所有元素类型相同,因为存在字符串''5'',所以7和12都被转换为character,当执行vector2{[}2{]}
    +
    vector2{[}3{]}时,试图对两个字符元素(即''7''和''12'')使用加法操作符+,在R中+不能用于character,因此会抛出异常。
  \item
    dataframe3 \textless-
    data.frame(z1=``5'',z2=7,z3=12)这句代码创建了一个有三列数据的dataframe,列名分别为z1,z2,z3,元素为''5'',7,12
    dataframe3{[}1,2{]} + dataframe3{[}1,3{]}
    将dataframe3第1行2列和3列的元素相加,返回19
  \item
    list4{[}{[}2{]}{]}+list4{[}{[}4{]}{]}返回结果为168,而list4{[}2{]}+list4{[}4{]}会报错,因为list4{[}{[}2{]}{]}返回的是列表list4中的第二个元素,可以直接运算;而list4{[}2{]}返回的是包含第二个元素的子列表,列表的数据类型不能进行运算。
  \end{itemize}
\item
  \begin{enumerate}
  \def\labelenumii{\alph{enumii}.}
  \tightlist
  \item
  \end{enumerate}

\begin{Shaded}
\begin{Highlighting}[]
 \CommentTok{\#design an expression that will give you the sequence of numbers from 1 to 10000 in increments of 372.}
\FunctionTok{seq}\NormalTok{(}\DecValTok{1}\NormalTok{,}\DecValTok{10000}\NormalTok{,}\AttributeTok{by=}\DecValTok{372}\NormalTok{)}
\end{Highlighting}
\end{Shaded}

\begin{verbatim}
##  [1]    1  373  745 1117 1489 1861 2233 2605 2977 3349 3721 4093 4465 4837 5209
## [16] 5581 5953 6325 6697 7069 7441 7813 8185 8557 8929 9301 9673
\end{verbatim}

\begin{Shaded}
\begin{Highlighting}[]
\CommentTok{\#Design another that will give you a sequence between 1 and 10000 that is exactly 50 numbers in length.}
\FunctionTok{seq}\NormalTok{(}\DecValTok{1}\NormalTok{,}\DecValTok{10000}\NormalTok{,}\AttributeTok{by=}\NormalTok{(}\DecValTok{10000}\SpecialCharTok{/}\DecValTok{50}\NormalTok{))}
\end{Highlighting}
\end{Shaded}

\begin{verbatim}
##  [1]    1  201  401  601  801 1001 1201 1401 1601 1801 2001 2201 2401 2601 2801
## [16] 3001 3201 3401 3601 3801 4001 4201 4401 4601 4801 5001 5201 5401 5601 5801
## [31] 6001 6201 6401 6601 6801 7001 7201 7401 7601 7801 8001 8201 8401 8601 8801
## [46] 9001 9201 9401 9601 9801
\end{verbatim}

  \begin{enumerate}
  \def\labelenumii{\alph{enumii}.}
  \setcounter{enumii}{1}
  \tightlist
  \item
    rep(1:3, times=3) 是将整个向量重复3次,输出为123123123;而rep(1:3,
    each=3)是将向量中每个元素重复三次,输出为111222333
  \end{enumerate}
\end{enumerate}

MB.Ch1.2.

\begin{Shaded}
\begin{Highlighting}[]
\FunctionTok{library}\NormalTok{(DAAG)}
\CommentTok{\#Create a new data frame by extracting these rows from orings}
\NormalTok{orings\_subset }\OtherTok{\textless{}{-}}\NormalTok{ orings[}\FunctionTok{c}\NormalTok{(}\DecValTok{1}\NormalTok{,}\DecValTok{2}\NormalTok{,}\DecValTok{4}\NormalTok{,}\DecValTok{11}\NormalTok{,}\DecValTok{13}\NormalTok{,}\DecValTok{18}\NormalTok{), ] }
\CommentTok{\#plot total incidents against temperature for this new data frame}
\FunctionTok{plot}\NormalTok{(orings\_subset}\SpecialCharTok{$}\NormalTok{Temperature,orings\_subset}\SpecialCharTok{$}\NormalTok{Total,}\AttributeTok{xlab =} \StringTok{"Temperature"}\NormalTok{,}\AttributeTok{ylab=}\StringTok{"Total Incidents"}\NormalTok{, }\AttributeTok{main =} \StringTok{"ORing Damage vs Temperature"}\NormalTok{)}
\end{Highlighting}
\end{Shaded}

\pandocbounded{\includegraphics[keepaspectratio]{homework1_files/figure-latex/unnamed-chunk-4-1.pdf}}

\begin{Shaded}
\begin{Highlighting}[]
\CommentTok{\#plot for the full dataset}
\FunctionTok{plot}\NormalTok{(orings}\SpecialCharTok{$}\NormalTok{Temperature,orings}\SpecialCharTok{$}\NormalTok{Total,}\AttributeTok{xlab =} \StringTok{"Temperature"}\NormalTok{,}\AttributeTok{ylab=}\StringTok{"Total Incidents"}\NormalTok{, }\AttributeTok{main =} \StringTok{"ORing Damage vs Temperature"}\NormalTok{)}
\end{Highlighting}
\end{Shaded}

\pandocbounded{\includegraphics[keepaspectratio]{homework1_files/figure-latex/unnamed-chunk-4-2.pdf}}

MB.Ch1.4. (a).

\begin{Shaded}
\begin{Highlighting}[]
\FunctionTok{str}\NormalTok{(ais) }
\end{Highlighting}
\end{Shaded}

\begin{verbatim}
## 'data.frame':    202 obs. of  13 variables:
##  $ rcc   : num  3.96 4.41 4.14 4.11 4.45 4.1 4.31 4.42 4.3 4.51 ...
##  $ wcc   : num  7.5 8.3 5 5.3 6.8 4.4 5.3 5.7 8.9 4.4 ...
##  $ hc    : num  37.5 38.2 36.4 37.3 41.5 37.4 39.6 39.9 41.1 41.6 ...
##  $ hg    : num  12.3 12.7 11.6 12.6 14 12.5 12.8 13.2 13.5 12.7 ...
##  $ ferr  : num  60 68 21 69 29 42 73 44 41 44 ...
##  $ bmi   : num  20.6 20.7 21.9 21.9 19 ...
##  $ ssf   : num  109.1 102.8 104.6 126.4 80.3 ...
##  $ pcBfat: num  19.8 21.3 19.9 23.7 17.6 ...
##  $ lbm   : num  63.3 58.5 55.4 57.2 53.2 ...
##  $ ht    : num  196 190 178 185 185 ...
##  $ wt    : num  78.9 74.4 69.1 74.9 64.6 63.7 75.2 62.3 66.5 62.9 ...
##  $ sex   : Factor w/ 2 levels "f","m": 1 1 1 1 1 1 1 1 1 1 ...
##  $ sport : Factor w/ 10 levels "B_Ball","Field",..: 1 1 1 1 1 1 1 1 1 1 ...
\end{verbatim}

\begin{Shaded}
\begin{Highlighting}[]
\FunctionTok{colSums}\NormalTok{(}\FunctionTok{is.na}\NormalTok{(ais))}\CommentTok{\#统计ais中每一列缺失值的数量,结果都为0,所以全都没有缺失值}
\end{Highlighting}
\end{Shaded}

\begin{verbatim}
##    rcc    wcc     hc     hg   ferr    bmi    ssf pcBfat    lbm     ht     wt 
##      0      0      0      0      0      0      0      0      0      0      0 
##    sex  sport 
##      0      0
\end{verbatim}

(b).

\begin{Shaded}
\begin{Highlighting}[]
\FunctionTok{library}\NormalTok{(dplyr)}
\end{Highlighting}
\end{Shaded}

\begin{verbatim}
## 
## Attaching package: 'dplyr'
\end{verbatim}

\begin{verbatim}
## The following objects are masked from 'package:stats':
## 
##     filter, lag
\end{verbatim}

\begin{verbatim}
## The following objects are masked from 'package:base':
## 
##     intersect, setdiff, setequal, union
\end{verbatim}

\begin{Shaded}
\begin{Highlighting}[]
\CommentTok{\#Make a table that shows the numbers of males and females for each different sport}
\NormalTok{sport\_sex\_table }\OtherTok{\textless{}{-}} \FunctionTok{table}\NormalTok{(ais}\SpecialCharTok{$}\NormalTok{sport, ais}\SpecialCharTok{$}\NormalTok{sex)}
\NormalTok{sport\_sex\_df }\OtherTok{\textless{}{-}} \FunctionTok{as.data.frame}\NormalTok{(sport\_sex\_table) }\CommentTok{\#将table转换为data frame}
\FunctionTok{colnames}\NormalTok{(sport\_sex\_df) }\OtherTok{\textless{}{-}} \FunctionTok{c}\NormalTok{(}\StringTok{"Sport"}\NormalTok{, }\StringTok{"Sex"}\NormalTok{, }\StringTok{"Count"}\NormalTok{) }\CommentTok{\#修改列名}
\NormalTok{grouped\_data }\OtherTok{\textless{}{-}} \FunctionTok{group\_by}\NormalTok{(sport\_sex\_df,Sport) }\CommentTok{\#将数据根据运动分组}
\CommentTok{\#统计}
\NormalTok{summarized\_data }\OtherTok{\textless{}{-}} \FunctionTok{summarise}\NormalTok{(}
\NormalTok{  grouped\_data,}
  \AttributeTok{Male =} \FunctionTok{sum}\NormalTok{(Count[Sex }\SpecialCharTok{==} \StringTok{"m"}\NormalTok{]),}
  \AttributeTok{Female =} \FunctionTok{sum}\NormalTok{(Count[Sex }\SpecialCharTok{==} \StringTok{"f"}\NormalTok{]),}
  \AttributeTok{Total =} \FunctionTok{sum}\NormalTok{(Count),}
  \AttributeTok{Ratio =} \FunctionTok{round}\NormalTok{(Male }\SpecialCharTok{/}\NormalTok{ Female, }\DecValTok{2}\NormalTok{)}
\NormalTok{)}
\CommentTok{\#筛选出男女比例大于2的运动}
\NormalTok{imbalanced\_sports }\OtherTok{\textless{}{-}} \FunctionTok{filter}\NormalTok{(summarized\_data,Ratio }\SpecialCharTok{\textgreater{}} \DecValTok{2} \SpecialCharTok{|}\NormalTok{ Ratio }\SpecialCharTok{\textless{}} \FloatTok{0.5} \SpecialCharTok{|} \FunctionTok{is.infinite}\NormalTok{(Ratio))}
\FunctionTok{print}\NormalTok{(imbalanced\_sports)}
\end{Highlighting}
\end{Shaded}

\begin{verbatim}
## # A tibble: 4 x 5
##   Sport    Male Female Total  Ratio
##   <fct>   <int>  <int> <int>  <dbl>
## 1 Gym         0      4     4   0   
## 2 Netball     0     23    23   0   
## 3 T_Sprnt    11      4    15   2.75
## 4 W_Polo     17      0    17 Inf
\end{verbatim}

\begin{Shaded}
\begin{Highlighting}[]
\CommentTok{\#或者使用管道运算符}
\NormalTok{imbalanced\_sports }\OtherTok{\textless{}{-}}\NormalTok{ sport\_sex\_df }\SpecialCharTok{\%\textgreater{}\%} \FunctionTok{group\_by}\NormalTok{(Sport) }\SpecialCharTok{\%\textgreater{}\%} \FunctionTok{summarise}\NormalTok{( }
  \AttributeTok{Male =} \FunctionTok{sum}\NormalTok{(Count[Sex }\SpecialCharTok{==} \StringTok{"m"}\NormalTok{]),}
  \AttributeTok{Female =} \FunctionTok{sum}\NormalTok{(Count[Sex }\SpecialCharTok{==} \StringTok{"f"}\NormalTok{]),}
  \AttributeTok{Total =} \FunctionTok{sum}\NormalTok{(Count),}
  \AttributeTok{Ratio =} \FunctionTok{round}\NormalTok{(Male }\SpecialCharTok{/}\NormalTok{ Female, }\DecValTok{2}\NormalTok{)) }\SpecialCharTok{\%\textgreater{}\%} \FunctionTok{filter}\NormalTok{(Ratio }\SpecialCharTok{\textgreater{}} \DecValTok{2} \SpecialCharTok{|}\NormalTok{ Ratio }\SpecialCharTok{\textless{}} \FloatTok{0.5} \SpecialCharTok{|} \FunctionTok{is.infinite}\NormalTok{(Ratio))}
\end{Highlighting}
\end{Shaded}

MB.Ch1.6.

\begin{Shaded}
\begin{Highlighting}[]
\NormalTok{ col1 }\OtherTok{\textless{}{-}} \FunctionTok{c}\NormalTok{(}\StringTok{"Winnipeg"}\NormalTok{,}\StringTok{"Winnipegosis"}\NormalTok{,}\StringTok{"Manitoba"}\NormalTok{,}\StringTok{"SouthernIndian"}\NormalTok{,}\StringTok{"Cedar"}\NormalTok{,}\StringTok{"Island"}\NormalTok{,}\StringTok{"Gods"}\NormalTok{,}\StringTok{"Cross"}\NormalTok{,}\StringTok{"Playgreen"}\NormalTok{)}
\NormalTok{ col2 }\OtherTok{\textless{}{-}} \FunctionTok{c}\NormalTok{(}\DecValTok{217}\NormalTok{,}\DecValTok{254}\NormalTok{,}\DecValTok{248}\NormalTok{,}\DecValTok{254}\NormalTok{,}\DecValTok{253}\NormalTok{,}\DecValTok{227}\NormalTok{,}\DecValTok{178}\NormalTok{,}\DecValTok{207}\NormalTok{,}\DecValTok{217}\NormalTok{)}
\NormalTok{ col3 }\OtherTok{\textless{}{-}} \FunctionTok{c}\NormalTok{(}\DecValTok{24387}\NormalTok{,}\DecValTok{5374}\NormalTok{,}\DecValTok{4624}\NormalTok{,}\DecValTok{2247}\NormalTok{,}\DecValTok{1353}\NormalTok{,}\DecValTok{1223}\NormalTok{,}\DecValTok{1151}\NormalTok{,}\DecValTok{755}\NormalTok{,}\DecValTok{657}\NormalTok{)}
\NormalTok{ Manitoba.lakes }\OtherTok{\textless{}{-}} \FunctionTok{data.frame}\NormalTok{(col2, col3) }\CommentTok{\#生成data frame}
\NormalTok{ Manitoba.lakes }\OtherTok{\textless{}{-}} \FunctionTok{data.frame}\NormalTok{(}\StringTok{"elevation"} \OtherTok{=}\NormalTok{ col2,}\StringTok{"area"} \OtherTok{=}\NormalTok{ col3) }\CommentTok{\#设置列名}
 \FunctionTok{row.names}\NormalTok{(Manitoba.lakes) }\OtherTok{\textless{}{-}}\NormalTok{ col1 }\CommentTok{\#设置行名}
\end{Highlighting}
\end{Shaded}

MB.Ch1.7. (a)

\begin{Shaded}
\begin{Highlighting}[]
\FunctionTok{dotchart}\NormalTok{(Manitoba.lakes}\SpecialCharTok{$}\NormalTok{area, }\AttributeTok{labels =} \FunctionTok{row.names}\NormalTok{(Manitoba.lakes), }\AttributeTok{main =} \StringTok{"Area of Manitoba Lakes (Linear Scale)"}\NormalTok{,}\AttributeTok{xlab =} \StringTok{"Area (square km)"}\NormalTok{)}
\end{Highlighting}
\end{Shaded}

\pandocbounded{\includegraphics[keepaspectratio]{homework1_files/figure-latex/unnamed-chunk-8-1.pdf}}

\begin{enumerate}
\def\labelenumi{(\alph{enumi})}
\setcounter{enumi}{1}
\tightlist
\item
\end{enumerate}

\begin{Shaded}
\begin{Highlighting}[]
 \FunctionTok{dotchart}\NormalTok{(}\FunctionTok{log2}\NormalTok{(Manitoba.lakes}\SpecialCharTok{$}\NormalTok{area), }\AttributeTok{labels =} \FunctionTok{row.names}\NormalTok{(Manitoba.lakes),}\AttributeTok{main =} \StringTok{"Area of Manitoba Lakes (Logarithmic Scale {-} log2)"}\NormalTok{, }\AttributeTok{xlab =} \FunctionTok{expression}\NormalTok{(log[}\DecValTok{2}\NormalTok{](Area) }\SpecialCharTok{\textasciitilde{}}\NormalTok{ (km}\SpecialCharTok{\^{}}\DecValTok{2}\NormalTok{)))}
\end{Highlighting}
\end{Shaded}

\pandocbounded{\includegraphics[keepaspectratio]{homework1_files/figure-latex/unnamed-chunk-9-1.pdf}}

MB.Ch1.8.

\begin{Shaded}
\begin{Highlighting}[]
\CommentTok{\#计算了所有列出的湖泊面积之和,作为Manitoba被水覆盖面积的下界}
\NormalTok{total\_water\_area }\OtherTok{\textless{}{-}} \FunctionTok{sum}\NormalTok{(Manitoba.lakes}\SpecialCharTok{$}\NormalTok{area)}
\end{Highlighting}
\end{Shaded}


\end{document}
